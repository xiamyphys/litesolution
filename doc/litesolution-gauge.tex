\chapter*{\pkg{LiteSolution}~试卷解析模板排版规范}% standard
\fancyhead[L]{\,\color{1号色}\kaishu\href{http://weixin.qq.com/r/hR1SSofEIdpercMp90iX}{物理問題作}}
\fancyhead[R]{\color{1号色}\ttfamily\mailto{xiamyphys@gmail.com}\,}


\centerline{夏明宇, \href{https://www.hdu.edu.cn}{杭州电子科技大学}}
\yyyymmdddate
\centerline{\mailto{xiamyphys@gmail.com}}

\section{模板介绍}
本模板用于高校期中 \& 期末试卷排版,支持一键输出无答案试卷.

\section{数学公式}
\subsection{正体问题}
\begin{enumerate}
	\item 微分算符要使用正体:\verb|\d|.
	\item 自然对数要使用正体:\verb|\e|.
	\item 三角函数、exp正体:\verb|\function|
	\item 撇\verb|'|和\verb|^\prime|的作用相同:\href{https://tex.stackexchange.com/a/538413/299948}{TeX Stackexchange}.
\end{enumerate}

\subsection{行内 \& 行间公式}
\begin{enumerate}
	\item 行内公式:要\verb|$ ... $|不要\verb|\( ... \)|
	\item 行间公式:要\verb|\[ ... \]|不要\verb|$$ ... $$|
\end{enumerate}
至于为何,“不要”的命令在\LaTeX 中属于「脆弱」命令,这里不做详解.

\href{https://www.zhihu.com/question/27589739}{\LaTeX 中的「要」和「不要」 - 知乎}

\subsection{选填题}
\begin{enumerate}
	\item 使用tasks环境判断高度
	\begin{verbatim}
		\begin{tasks}(一行选项个数,自己掂量)
			\task 选项1 \task 选项2
			\task 选项3 \task 选项4
		\end{tasks}
	\end{verbatim}

	不要使用\verb|tabular|制作选择题选项,选项不是画表格!
	\item 答案不要擅自使用\verb|\underline{}|命令,而是使用\verb|\ans{}|命令,否则制作空白无答案版时答案将无法隐藏.
\end{enumerate}

\subsection{等高括号}
\begin{enumerate}
	\item 凡是你觉得很高的都需做等高括号处理:\verb|\ab( ), \ab[ ], \ab\{ \}|
	\item 但也不要所有括号都做等高处理,比如\verb|f\ab(x)|和\verb|f(x)|的效果是一样的,多一个\verb|\ab|判断高度只会徒增编译时间.
\end{enumerate}

\section{偷懒技巧}
以\verb|\frac{}{}|为例
\begin{enumerate}
	\item \verb|\frac12 == \frac{1}{2}, \frac\pi2 == \frac{\pi}{2}, \frac yx == \frac{y}{x}|
	\item \verb|frac{}{}|后跟两个键值,如果未使用\verb|{}|界定作用域,则默认scan到第1个char或宏(如\verb|\alpha|)时将其放入第一个键值,scan到第2个char或宏(如\verb|\alpha|)时将其放入第2个键值.
	\item 不会用的话别乱用,老实儿使用\verb|{}|界定作用域,不要出现诸如\verb|\frac x'2|: $\frac x'2$(事实你要写的是$\frac{x'}{2}$).
\end{enumerate}

\section{排版美观}
\begin{enumerate}
	\item 尽量不要(甚至杜绝)题目在上一页,答案在下一页的情况(\emph{\textbf{六级完形文章和题目不在一页的痛你应该经历过}}).
	\item 杜绝出现题目/解答盒子断开跨页的情况,这很不美观. 如果一个题目解析实在过长,请尝试缩句,或者使用分栏\verb|multicols|环境以节省空间.
	\item 如果因第$n+1$页开头的题目盒子高度小于第$n$页末尾剩余了空间而导致了空间浪费,请试着将后文中高度与第$n$页最后一道题目+剩余空间相当的题目位置与其交换.
	\item 最完美的排版:除非遇到下一个\verb|section|,每一页从头到尾都恰好填满了题目与解析.
\end{enumerate}